\documentclass[10pt]{article}
\usepackage[english]{babel}

\usepackage{mathptmx}
\usepackage[table, rgb]{xcolor}
\usepackage{enumerate}
\usepackage[bottom=1in]{geometry}
\usepackage{graphicx}
\usepackage{titlesec}
\usepackage{fancyhdr}
\usepackage{lastpage}
\usepackage{array}
\usepackage{booktabs}
\usepackage{ltablex}
\setlength{\parskip}{1em}

\pagestyle{fancy}
\renewcommand{\headrulewidth}{0pt} % % remove top line
\cfoot{\thepage\ / \pageref{LastPage}}

\newlength{\interwordspace}
\settowidth{\interwordspace}{\ }
% %%%%%%%%%%%%%%%%%%%%%%%%%%%%%%%%%%%%%%%%%%%%%%%%%%%%%%%%%%%%%%%%%%%%%%%%%%%%%%%
% %%%%%%%%%%%%%%%%%%%%%%%%%%%%%%%%%%%%%%%%%%%%%%%%%%%%%%%%%%%%%%%%%%%%%%%%%%%%%%%
% %%%%%%%%%%%%%%%%  The following command only exist for development %%%%%%%%%%%%
% %%%%%%%%%%%%%%%%  In production, JinJa will have gotten rid of     %%%%%%%%%%%%
% %%%%%%%%%%%%%%%%  the VAR and BLOCK commands.                      %%%%%%%%%%%%
% %%%%%%%%%%%%%%%%%%%%%%%%%%%%%%%%%%%%%%%%%%%%%%%%%%%%%%%%%%%%%%%%%%%%%%%%%%%%%%%
\makeatletter
\newcommand{\VAR}[1]{{\color{blue} #1}}
\newcommand{\BLOCK}[1]{~\newline{ \color{red}#1 }}
\makeatother


% For stuff that only appears in html version and should be verbatim
% This works because HCode is defined for htlatex but not for pdflatex,
% hence providecommand overrides the command for pdflatex but for htlatex it is
% the normal HCode command, which writes html code verbatim
\providecommand{\HCode}[1]{}
% to group elements into a div in html
\newenvironment{htmldiv}{}{}


%[ if toPdf
\usepackage[T1]{fontenc}
\usepackage[scaled]{helvet}
%[ endif
\renewcommand*\familydefault{\sfdefault}

\graphicspath{{files/}}

\definecolor{slightlylightergray}{rgb}{0.93333,0.93,0.9137}

\titlespacing\subsection{0pt}{0pt}{4pt}
\setlength\parindent{0pt} % no indentation for all document

%---------------------------------------------------------------------------------------------------------------------------------------------------------------------
\begin{document}


\begin{htmldiv}
    \begin{minipage}[t]{.5\textwidth}
        \BLOCK{ if logo is defined }
        \strut\vspace*{-\baselineskip}\newline % this line is needed to align minipages top
            \begin{minipage}{.25\textwidth}
                \BLOCK{if toPdf}
                \includegraphics[width=\textwidth]{logo.png}
                \BLOCK{ else }
                \HCode{<img alt="PIC" src="\VAR{logo}"/>}
                \BLOCK{ endif }
            \end{minipage}
            \begin{minipage}{.23\textwidth}
                \begin{tabular}{@{}ll}
                    \VAR{ senderName }  \\ \VAR{ senderAddress } \\ \VAR{ senderCity }, \VAR{ senderStateInitials }, \VAR{ senderZipCode }
                \end{tabular}
            \end{minipage}
        \BLOCK{ else }
        \begin{minipage}[t]{.23\textwidth}
            \strut\vspace*{-\baselineskip}\newline % this line is needed to align minipages top
            \begin{tabular}{@{}ll}
                \VAR{ senderName }  \\ \VAR{ senderAddress } \\ \VAR{ senderCity }, \VAR{ senderStateInitials }, \VAR{ senderZipCode }
            \end{tabular}
        \end{minipage}
        \BLOCK{ endif }
        \vspace{10px}\\
        Bill to:\vspace{6px}\\
        \begin{tabular}{@{}l}
            \textbf{\VAR{recipientName }} \\
            \textbf{\VAR{recipientAddress } }\\
            \textbf{\VAR{recipientCity }, \VAR{ recipientStateInitials } \VAR{ recipientZipCode } }
        \end{tabular}
    \end{minipage}\hfill
    \begin{minipage}[t]{.5\textwidth}
        \begin{flushright}
            \strut\vspace*{-\baselineskip}\newline % this line is needed to align minipages top
            \begin{tabular}{ll}
                \multicolumn{2}{l}{\textbf{\begin{Huge}INVOICE  \end{Huge}}}\\
                \addlinespace[0px]\\
                Inv\#: & \VAR{ invoiceNumber }\\
                Issued: & \VAR{ issuedDate }\\
                Due: & \VAR{ dueDate }
            \end{tabular}
        \end{flushright}
    \end{minipage}
\end{htmldiv}




\vspace{12px}
\def\arraystretch{1.5} % For vertical spacing in table
% The table is build so that the Date, Units/hrs, Rate and Subtotal columns get the size they need. The remaining width
% is then split between Item name and Description. Where item name gets 35% (.7) and Description gets 65%(1.3) of the remaining space
\begin{tabularx}{\linewidth}[l]{l>{\hsize=.7\hsize\linewidth=\hsize}X>{\hsize=1.3\hsize\linewidth=\hsize}Xlrr}
    \cellcolor{slightlylightergray}  Date & \cellcolor{slightlylightergray} Item name &  \cellcolor{slightlylightergray}Description &  \cellcolor{slightlylightergray}Units/hrs & \cellcolor{slightlylightergray} Rate & \cellcolor{slightlylightergray} Subtotal \\
    \endhead
        %[ for entry in entries
        \VAR{ entry.date } & \VAR{ entry.title } & \VAR{ entry.description } & \VAR{ entry.units } & \$\VAR{ entry.rate } & \$\VAR{ entry.total }\\
        %[ endfor
        \cline{1-6}
\end{tabularx}
\vspace{9px}
\begin{flushright}\begin{tabular}{rrr}
        \textbf{Subtotal} && \textbf{\$\VAR{ subtotal }}\\
        \BLOCK{if taxesAmount is defined }
        Taxes & \VAR{ taxesPercent }\%  & \$\VAR{ taxesAmount }\\
        \BLOCK{ endif }
        Discount && -\$\VAR{ discountAmount }\\
        \cline{1-3} \vspace{-12px}\\
        \begin{large}\textbf{Total} \end{large} && \begin{large} \textbf{\$\VAR{ totalAmount }}\end{large}\\
    \end{tabular}
\end{flushright}
\vspace{1cm}


\subsection*{\color{lightgray} Late Fees}
If this invoice is unpaid by \VAR{ dueDate }, a non-compounding late fee of \VAR{ feePercentage }\% accrues monthly on the outstanding amount.

\BLOCK{ if notes is defined }
\vspace{1cm}
\subsection*{\color{lightgray} Notes}
\VAR{ notes }
\BLOCK{ endif }

\end{document}